\section{Geometrical Algorithms}

\subsection{Dot Product}

\begin{verbatim}
int dot(int[] A, int[] B, int[] C) {
   int AB[2], BC[2];
   AB[0] = B[0]-A[0];
   AB[1] = B[1]-A[1];
   BC[0] = C[0]-B[0];
   BC[1] = C[1]-B[1];
   int dot = AB[0] * BC[0] + AB[1] * BC[1];
   return dot;
}
\end{verbatim}

\subsection{Cross Product}
\begin{verbatim}
int cross(int[] A, int[] B, int[] C) {
   int AB[2], AC[2];
   AB[0] = B[0]-A[0];
   AB[1] = B[1]-A[1];
   AC[0] = C[0]-A[0];
   AC[1] = C[1]-A[1];
   int cross = AB[0] * AC[1] - AB[1] * AC[0];
   return cross;
}
\end{verbatim}

\subsection{Point on segment}

A point is on a segment if its distance to the segment is 0.

\textbf{Given two different points $(x_1,y_1)$ and $(x_2,y_2)$ the values of $A$,$B$, and $C$ for $Ax+By+C=0$ are given by}\\
$$
A=y_2-y_1
$$
$$
B=x_1-x_2\\
$$
$$
C=A*x_1+B*y_1\\
$$

\subsection{Intersection of segments}
\begin{verbatim}
double det = A1*B2 - A2*B1
if (det == 0) {
   //Lines are parallel
} else {
   double x = -(A1*C2 - A2*C1) / det
   double y = -(B1*C2 - B2*C1) / det
}
\end{verbatim}

\subsection{Posi��o de ponto em rela��o � recta}
\begin{verbatim}
//Input:  three points P0, P1, and P2
//Return: >0 for P2 left of the line through P0 and P1
//        = 0 for P2 on the line
//        < 0 for P2 right of the line
int isLeft( Point P0, Point P1, Point P2 ) {
    return ( (P1.x - P0.x) * (P2.y - P0.y)
            - (P2.x - P0.x) * (P1.y - P0.y) );
}
\end{verbatim}

\subsection{Dist�ncia entre um ponto e uma recta ou segmento de recta}

If the line is in the form $Ax+By+C=0$:
$$
d=\frac{|Ax_0+By_0+C|}{\sqrt{A^2+B^2}}
$$

\begin{verbatim}
//Compute the dist. from AB to C
//if isSegment=strue, AB is a seg., not a line.
double linePointDist(int[] A, int[] B,
             int[] C, boolean isSegment) {
   double dist = cross(A,B,C) / distance(A,B);
   if (isSegment) {
      int dot1 = dot(A,B,C);
      if (dot1 > 0) return distance(B,C);
      int dot2 = dot(B,A,C);
      if (dot2 > 0) return distance(A,C);
   }
   return abs(dist);
}
\end{verbatim}

\subsection{Polygon Area}
\begin{verbatim}
int area = 0;
/*int N = lengthof(p);*/

for (int i = 1; i + 1 < N; i++) {
   int x1 = p[i][0] - p[0][0];
   int y1 = p[i][1] - p[0][1];
   int x2 = p[i+1][0] - p[0][0];
   int y2 = p[i+1][1] - p[0][1];
   int cross = x1*y2 - x2*y1;
   area += cross;
}
return fabs(area/2.0);
\end{verbatim}

\subsection{Convex Hull} 
\begin{verbatim}
// Devolve a lista de pontos no convex hull na
// ordem contr�ria aos rel�gios.
// Nota: o �ltimo ponto na lista devolvida � 1 
// replica��o do primeiro
#include <vector>
vector<point> ConvexHull(vector<point> P) {
       int n = P.size(), k = 0;
       vector<point> H(2*n);

       // Sort points lexicographically
       sort(P.begin(), P.end());

       // Build lower hull
       for (int i = 0; i < n; i++) {
               while (k >= 2 && 
                      cross(H[k-2], H[k-1], P[i]) <= 0) 
                  k--;
               H[k++] = P[i];
       }

       // Build upper hull
       for (int i = n-2, t = k+1; i >= 0; i--) {
               while (k >= t && 
               cross(H[k-2], H[k-1], P[i]) <= 0)
                  k--;
               H[k++] = P[i];
       }

       H.resize(k);
       return H;
}
\end{verbatim}

\subsection{Closest pair of points}
\begin{verbatim}
double delta_m(vp &ql, vp &qr,double delta) {
   int i,j=0;
   double dm=delta;
   for (i=0;i<(int)ql.size();i++) {
      point p=ql[i];
      
      while (j<(int)qr.size() && qr[j].y < p.y-delta)
         j++;
      
      int k=j;
      while (k<(int)qr.size()
             && qr[k].y<=p.y+delta) {
         dm=min(dm,dist(p,qr[k]));
         k++;
      }
   }
   return dm;
}
	    

vp select_candidates(vp &p,int l,int r,
               double delta,double midx) {
   vp n;
   for (int i=l;i<=r;i++) {
      if (abs(p[i].x-midx)<=delta)
         n.push_back(p[i]);
   }
   return n;
}

double closest_pair(vp &p,int l,int r) {
    if (r-l+1<2)return INF;
    int mid=(l+r)/2;
    double midx=p[mid].x;
    double dl=closest_pair(p,l,mid);
    double dr=closest_pair(p,mid+1,r);
    double delta=min(dl,dr);

    vp ql,qr;
    ql=select_candidates(p,l,mid,delta,midx);
    qr=select_candidates(p,mid+1,r,delta,midx);

    double dm=delta_m(ql,qr,delta);

    vp res;
    merge(p.begin()+l,p.begin()+mid+1,
          p.begin()+mid+1,p.begin()+r+1,
               back_inserter(res),cmp);
    copy(res.begin(),res.end(),p.begin()+l);
    return min(dm,min(dr,dm));
}
\end{verbatim}

\subsection{Test if point is inside a polygon}
\begin{verbatim}
//Input: P = ponto
//       V[] = vector de n+1 pontos, com V[n]=V[0]
//Return:  wn = 0 only if P is outside V[]
int wn_PnPoly( Point P, Point* V, int n ) {
    int    wn = 0;    // the winding number counter

    // loop through all edges of the polygon
    for (int i=0; i<n; i++) {
        if (V[i].y <= P.y) {
            if (V[i+1].y > P.y)
                if (isLeft( V[i], V[i+1], P) > 0)
                    ++wn;
        }
        else {
            if (V[i+1].y <= P.y)
                if (isLeft( V[i], V[i+1], P) < 0)
                    --wn;
        }
    }
    return wn;
}
\end{verbatim}
\subsection{Intersection of rectangles}
\begin{verbatim}
x1 = y2 = INT_MIN;
x2 = y1 = INT_MAX;

for (i=0;i<NRECT;i++) { //for all rectangles
   cin>>ax>>ay>>bx>>by;
   if (ax>x1)x1=ax;
   if (ay<y1)y1=ay;
   if (bx<x2)x2=bx;
   if (by>y2)y2=by;
}
\end{verbatim}

\subsection{Circle from 3 points}
\begin{verbatim}
int main() {
   double ax,ay,bx,by,cx,cy,xres,yres;
   double xmid,ymid,A1,B1,C1,A2,C2,B2,dist;
   
   while (scanf("%lf %lf %lf %lf %lf %lf",
                    &ax,&ay,&bx,&by,&cx,&cy)==6) {
      A1 = by - ay;
      B1 = ax - bx;
      xmid=min(ax,bx)+(max(ax,bx)-min(ax,bx))/2.0;
      ymid=min(ay,by)+(max(ay,by)-min(ay,by))/2.0;
      C1 = -B1 * xmid + A1 * ymid;
      
      B2 = bx - cx;
      A2 = cy - by;
      xmid=min(bx,cx)+(max(bx,cx)-min(bx,cx))/2.0;
      ymid=min(by,cy)+(max(by,cy)-min(by,cy))/2.0;
      C2 = -B2 * xmid + A2 * ymid;
      //intersection of segments
      intersection(A1,B1,C1,A2,B2,C2,&xres,&yres);
      dist = sqrt(pow(xres-bx,2)+pow(yres-by,2));
      printf("(x %s %.3lf)^2 + (y %s %.3lf)^2 =
      %.3lf^2\n",xres<0.0?"+":"-",abs(xres),
               yres<0.0?"+":"-",abs(yres),dist);
   }

    return 0;
}
\end{verbatim}
