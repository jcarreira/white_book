\section{Strings}

\subsection{KMP}
\begin{verbatim}
int* compute_prefix(string p) {
    int m=p.length();
    int *pi=(int*)malloc(sizeof(int)*m);
    pi[0]=0;
    int k=0;
    for(int q=1;q<m;q++) {
        while(k>0&&p[k]!=p[q])
            k=pi[k-1];
        if(p[k]==p[q])
            k++;
        pi[q]=k;
    }
    return pi;
}

void kmp_match(string s,string p) {
    int *pi=compute_prefix(p);
    int q=0;
    int n=s.length();
    int m=p.length();

    for(int i=0;i<n;i++) {
        while(q>0&&p[q]!=s[i])
            q=pi[q-1];
        if(p[q]==s[i])
            q++;
        if(q==m) {
            printf("Match pos %d\n",i-m+1);
        }
    }
}
\end{verbatim}

\subsection{Range Minimum Query}
\begin{verbatim}
int main() {
   int N,Q,i,j,k;

   scanf("%d %d",&N,&Q);

   for (i=0;i<N;i++)
	    scanf("%d",&n[i]);

   for (i=0;i<N;i++)
	    m[i][0]=M[i][0]=n[i];

   for (i=1;(1<<i)<=N;i++) {
	    for (j=0;j+(1<<i)-1<N;j++) {
	       m[j][i]=min(m[j][i-1],m[j+(1<<(i-1))][i-1]);
	       M[j][i]=max(M[j][i-1],M[j+(1<<(i-1))][i-1]);
	    }
   }

   for (k=0;k<Q;k++) {
	    scanf("%d %d",&i,&j);
	    i--;j--;
	    int t,p;
	    t=(int)(log(j-i+1)/log(2));
	    p=1<<t;
	    printf("%d\n",max(M[i][t],M[j-p+1][t])
	    - min(m[i][t], m[j - p + 1][t]));
   }
   return 0;
}
\end{verbatim}
$$
M[i][j]=\max(M[i][j-1],M[i+2^{j-1}][j-1])
$$
$$
RMQ_A(i,j)=\max(M[i][k],M[j-2^k+1][k])
$$

\subsection{Nth Permutation}
\begin{verbatim}
/**
  * Computes kth (0 to s.size()! - 1) permutation
  * of string s
  */
std::string nth_permutation(uint64_t k, const std::string &s) {
    uint64_t factorial = 1;
    for (uint64_t i = 1; i <= s.size(); ++i) {
        factorial *= i;
    }

    std::string s_copy = s;
    std::string res;
    for (uint64_t j = 0; j < s.size(); ++j) {
        // compute how many permutations on the rest
        // of the string s[j + 1 .. s.size() - 1]
        factorial /= s.size() - j;
       
        // store character 
        uint64_t l = k / factorial;
        res += s_copy[l];

        // remove already used character
        s_copy.erase(s_copy.begin() + l);

        // compute new value of k
        k = k % factorial;
    }
    return res;
}
\end{verbatim}
