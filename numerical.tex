\section{Numerical}

\subsection{Choose}
$\binom{n}{k}$
\begin{verbatim}
long long memo[SIZE][SIZE];//initialized to -1
long long binom(int n,int k){
   if(memo[n][k]!=-1)return memo[n][k];
   if(n<k)return 0;
   if(n==k)return 1;
   if(k==0)return 1;
   return memo[n][k]=binom(n-1,k)+binom(n-1,k-1);
}
\end{verbatim}

\subsection{M�dulo:}
\begin{verbatim}
int mod(int a, int n) {
  return (a%n + n)%n;
}
\end{verbatim}

\subsection{LCM / GCD}
\begin{verbatim}
int gcd(int a,int b){
   if(!b)
      return a;
   else return gcd(b,a%b);
}

struct triple{
   int gcd,x,y;
   int triple(int g=0,int a=0,int b=0):
                       gcd(g),x(a),y(b){}
};

triple ExtendedEuclid(int a,int b){
   if(!b)
      return triple(a,1,0);
      
   triple t=ExtendedEuclid(b,a%b);
   return triple(t.gcd,t.y,t.x-(a/b)*t.y);
}

int LCM(int a,int b){
   return a*b/gcd(a,b);
}
   
\end{verbatim}


\subsection{Base conversion}
\begin{verbatim}
void base(char *res, int num, int base){
   char tmp[100];
   int i, j;
   for (i = 0; num; i++) {
      tmp[i]="0123456789ABCDEFGHIJKLM"[num%base];
      num /= base;
   }
   tmp[i] = 0;
   for (i--, j = 0; i >= 0; i--, j++)
      res[j] = tmp[i];
   res[j] = 0;
}
\end{verbatim}

\subsection{Horner's Rule}
$$
P(x)=\sum_{k=0}^{n}a_k x^k=a_0+x(a_1+x(a_2+\cdots+(a_{n-1}+x1_n)))
$$
\begin{verbatim}
double Horner(double coef[],int degree,int x) {
   double res = 0;
   int i;
   
   for (i = degree; i >= 0; i--)
      res = coef[i] + x * res;
   return res;
}
\end{verbatim}

\subsection{Big Mod}
$(B^P)\%M$
\begin{verbatim}
long int bigmod(long long int B,
         long long int P, long long int M) {
   if (P == 0)
      return 1;
   else if(P & 1) {
      long long int tmp = 
             bigmod(B,(P - 1) >> 1, M) % M;
	    tmp = (tmp * tmp * B) % M;
	    return tmp;
   }

   else{
      long long int tmp = bigmod(B, P >> 1, M) % M;
	    return (tmp * tmp) % M;
   }
}
\end{verbatim}

\subsection{Matrix Multiplication}
\begin{verbatim}
void Matrix_Multiply(int A[N][P],
                  int B[P][M],int N){
   int C[N][M],i,j,k;
   for(i=0;i<N;i++){
      for(j=0;j<P;j++){
         C[i][j]=0;
         for(k=0;k<P;k++)
            C[i][j]+=A[i][k]*B[k][j];
      }
   }
}
\end{verbatim}

\subsection{Long Arithmetic}

\textbf{Take care of leading zeroes.}

\textbf{Addition:}
\begin{verbatim}
/* make sure num1 and num2 are
   filled with '\0' after digits!! */
void add(char *num1,char *num2,char *res){
   int i,carry=0;
   reverse(num1,num1+strlen(num1));
   reverse(num2,num2+strlen(num2));
   
   for(i=0;num1[i] || num2[i];i++){
      res[i]=num1[i]+num2[i]-'0'+carry;
      if(!num1[i] || !num2[i])res[i]+='0';
      if(res[i]>'9'){
         carry=1;
         res[i]-=10;
      }else carry=0;
   }
   if(carry)res[i]='1';
   reverse(res,res+strlen(res));
}
\end{verbatim}

\textbf{Multiplication}
\begin{verbatim}
void mul(char *num1,char *num2,char *str){
   int i,j,res[2*SIZE]={0},carry=0;
   
   reverse(num1,num1+strlen(num1));
   reverse(num2,num2+strlen(num2));
   for(i=0;num1[i];i++)
      for(j=0;num2[j];j++)
         res[i+j]+=(num1[i]-'0')*(num2[j]-'0');
   for(i=2*SIZE-1;i>=0 && !res[i];i--);
   
   if(i<0)strcpy(str,"0");return;
   for(j=0;i>=0;i--,j++){
      str[j]=res[i]+carry;
      carry=str[j]/10;
      str[j]%=10;
      str[j]+='0';
   }
   if(carry)str[j]=carry+'0';
}
\end{verbatim}

\subsection{Infix para Postfix}
\begin{verbatim}

#define oper(a) ((a) == '+' || (a) == '-' \\
	|| (a) == '*' || (a) == '/')

// true if either:  !!
// b is left associative and
// its precedence is <= than a
//
// b is right associative and
// its prec is < than a
bool be_prec(char a, char b) {
   int p[300];
   p['+'] = p['-'] = 1;
   p['*'] = p['/'] = 2;
   return p[a] >= p[b];
}

string shunting_yard(string exp) {
    int i = 0;
    string res;
    stack<char> s; //operators (1 char!)

    while (i < exp.size()) {
	// if it is a function token push it onto the stack

	// If it is a func arg separator (e.g., a comma):
	// Until the topmost elem of the stack is '('
	// pop the elem from the stack and
	// append it to res. If no '(' -> error
	// do not pop '('

	if (isdigit(exp[i]) || exp[i] == 'x') { 
	    //number. add isalpha() for vars
	    for (; i < exp.size() &&
		    (isdigit(exp[i]) || exp[i] == 'x'); i++)
		res.push_back(exp[i]);
	    res.push_back(' ');
	    i--; //there's a i++ down there
	}
	else if (exp[i] == '(') s.push('(');

	else if (exp[i] == ')') {
	    while (!s.empty() && s.top() != '(') {
		res += (s.top() + string(" "));
		s.pop();
	    }
	    if (s.top() != '(') ;//error
	    else s.pop();
	}
	else if (oper(exp[i])) { //operator
	    while (!s.empty() && oper(s.top()) &&
		    be_prec(s.top(), exp[i])) {
		res += (s.top() + string(" "));
		s.pop();
	    }
	    s.push(exp[i]);
	}
	i++;
    }
    while (!s.empty()) {
	if (s.top() == '(' || s.top() == ')') ;//error
	res += (s.top() + string(" "));
	s.pop();
    }
    if (*(res.end() - 1) == ' ') res.erase(res.end() - 1);
    return res;
}


\end{verbatim}

\subsection{Calculate Postfix expression}
\begin{verbatim}
// exp is in postfix
double calc(string exp) {
    stack<double> s;
    istringstream iss(exp);
    string op;

    while (iss >> op) {
        // ATTENTION TO THIS
        if (op.size() == 1 && oper(op[0])) {
            if (s.size() < 2) ;//error
            double a = s.top(); s.pop();
            double b  = s.top(); s.pop();
            switch (op[0]) {
                case '+': s.push(b + a); break;
                case '-': s.push(b - a); break;
                case '*': s.push(b * a); break;
                case '/': s.push(b / a); break;
            }
        } else {
            istringstream iss2(op);
            double tmp;
            iss2 >> tmp;
            s.push(tmp);
        }
    }
    return s.top();
}
\end{verbatim}

\subsection{Postfix to Infix}
\begin{verbatim}
/*
 * Pass a stack with the expression to rpn2infix.
 * Ex: (bottom) 3 4 5 * + (top)
 */
string rpn2infix(stack<string> &s) {
   string x = s.top();
   s.pop();
   if(isdigit(x[0])) return x;
   else return string("(") +
    rpn2infix(s) + x + rpn2infix(s) + string(")");
}
\end{verbatim}

\subsection{Matrix Multiplication}
$$
C_{ij}=\sum_{k=1}^n a_{ik}.b_{kj}
$$

\begin{verbatim}
void matrix_mul(int A[N][P],int B[P][M]) {
   int C[N][M],i,j,k;
   for (i=0;i<N;i++) {
      for (j=0;j<P;j++) {
         C[i][j]=0;
         for (k=0;k<P;k++)
            C[i][j]+=A[i][k]*B[k][j];
      }
   }
}
\end{verbatim}


\subsection{Catalan Numbers}

$$
C_n = \frac{(2n)!}{(n+1)!n!}
$$

\begin{itemize}
\item $C_n$ counts the number of expressions containing $n$ pairs of parentheses which are correctly matched
\item $C_n$ is the number of different ways a convex polygon with $n + 2$ sides can be cut into triangles by connecting vertices with straight lines.
\end{itemize}

\subsection{Fibonnaci}
\begin{verbatim}
long fib(long n){
   long matrix[2][2]={{1,1},{1,0}};
   long res[2][2]={{1,1},{1,0}};
   while(n){
      if(n&1) /* se n e impar*/
         matrix_mul(matrix,res,res);
      matrix_mul(matrix,matrix,matrix);
      n/=2;
   }
   return res[1][1];
}
\end{verbatim}

{\huge Stuff}

\begin{itemize}
\item $(1+2+3+\cdot+n)^2=\big(\frac{n*(n+1)}{2}\big)^2=\frac{n^2*(n+1)^2}{4}$

\item $1^2+2^2+3^3+\cdot+n^2=\frac{n*(n+1)*(2n+1)}{6}$

\item $x$ is a power of two iff \begin{verbatim}(x & (x-1))==0\end{verbatim}.\\

\item In a quadratic
$$
Ax^2+Bx+C=0
$$the sum of the roots is $\frac{-B}{A}$ and the product of the roots is the constant term $C$.
\item The division of 2 complex numbers $w$ and $z$ is $\frac{z}{w}=\frac{z. w^{-1}}{w. w^{-1}}$ where $z^{-1}$ is the complex conjugate of $z$. $(a+bi)^{-1}=(a-bi)$\\\\
$(x+y)^n=\sum_{i=0}^{n}\binom{n}{i}x^{n-i} y^i$\\\\
$\binom{n}{m} = \frac{n!}{m!(n-m)!}$\\\\
$\binom{n}{0}+\binom{n}{1}+\binom{n}{2}+\cdots+\binom{n}{n}=2^n$\\\\
$\binom{n}{0}-\binom{n}{1}+\binom{n}{2}-\cdots+(-1)^n\binom{n}{n}=0$\\\\
$\binom{n}{k}=\binom{n-1}{k}+\binom{n-1}{k-1}$\\\\
For positive integers $n$,$t$, the coefficient of ${x_1}^{n_1}{x_2}^{n_2}{x_3}^{n_3}\cdots{x_t}^{n_t}$ in the expansion of $(x_1+x_2+x_3+\cdots+x_t)^n$  is
$$
\binom{n}{n_1}\binom{n-n_1}{n_2}\cdots\binom{n-n_1-n_2-n_3-\cdots-n_{t-1}}{n_t}=
$$
$$
\frac{n!}{n_1!n_2!n_3!\cdots n_t!}
$$

where each $n_i$ is an integer with $0\le n_i\le n$,for all $1\le i\le t$,and $n_1+n_2+n_3\cdots n_t=n$.\\

\end{itemize}

\subsection{Set Partitions}

$B_n$ is the number of differente partitions of a set with $n$ elements.
$B_0=1, B_1=1, B_2=2, B_3=5, B_4=15$

$B_{n+1}=\sum_{k=0}^n{\binom{n}{k}B_k}$

$S(n,k)$ is the number of ways to partition a set of $n$ elements in $k$ nonempty subsets.
$S(n,k) = S(n-1,k-1)+kS(n-1,k)$
